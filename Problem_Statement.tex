\documentclass[draftclsnofoot, onecolumn, 10pt]{IEEEtran}

\title{\huge HP Big Data Analytics: Effects of Memory Management and Parallelism on Query Performance}

\author{Oregon State University\\Senior Design: 2016-2017\\\\Prepared By:\\Nic Desilets, Nathaniel Whitlock,\\ and James Stallkamp}

\usepackage{titling}

\usepackage{geometry}

\usepackage{graphicx}

\parindent=0.0in

\parskip=0.2in

\begin{document}
\begin{titlingpage}
    \maketitle
    
%%    \vspace*{1cm} 
    \begin{abstract}
		Computational efficiency and performance are paramount to being able to quickly
		and effectively analyze data within a reasonable time frame when working with
		very large data sets. Oracle 12c, a cloud enabled relational database, performs
		sub-optimally by default when it comes to analyzing the massive amounts of
		telemetry data generated from HP PageWide Web Press printers. In order to achieve
		better performance from Oracle 12c, we experimented with various memory management
		methods in order to maximize RAM usage during a query. We also modified database
		table physical design parameters such as partitioning and measured the effects on
		parallel processing.  Additionally, we developed a toolkit alongside this project
		in order to accurately measure and compare the efficiency of different methods of
		memory management and parallelism on database queries. Using this toolkit, we have
		found the most efficient modifications to the relational database to be … based
		on quantitative comparative analysis of partition designs.
    \end{abstract}
\end{titlingpage}

%% Potentially add figure here

\section*{Problem Definition}
Each and every day, HP’s PageWide Web Presses generate roughly 250GB of data.
The data generated is normalized and loaded into databases in order to store
it so it can be queried for troubleshooting and performing engineering analytics.
The current database configuration only utilizes a small portion of the available
72 processor cores and 3TB of RAM it has available. There are plenty of physical
resources available, yet the current database configuration tends to under
allocate memory and CPU usage for large queries. When the Programming Global Area
(PGA) reaches a memory threshold the process much reach out to disk in order to
complete. Reading or writing to disk is one of the most costly operations for a
program. Additionally, there are instances where a query process is allocated a
large amount of cores to work with but when the processor use ratio is evaluated
it shows that only a couple of cores were used on average.

The under allocation of resources leads to longer periods of wall time during
data aggregation and prevents accessibility to real time performance metrics. By
experimenting with database configuration settings and table partition designs, we
hope to reduce the wall time observed during routine analytics.



\section*{Problem Solution}
In short the problem is that the server is responding very slowly, despite having
idle resources. Our goal is to help improve resource usage of the server, and to 
improve the performance efficiency index. To do this we will develop a toolkit that
will allow users to examine information pertaining to the performance specs of the
query. This toolkit will consist of PL/SQL statements which will query the dynamic
performance view to generate a report. Each report will represent the performance
results of a query of the database. Every report will serve as the unique profile
of an experimental design that was set up in order to test the performance of the
physical hardware. These reports will help users to adjust settings on the server
and in queries to improve resource usage and runtime. Our metric for success will
be to improve the performance efficiency index, which is the database run time / real time.
Specifically we would like the queries to be able to run fast enough to run in real
time, the client specified 5 seconds or less. The results we will take away from this
will be data showing the improvement gained. The data will show how much we increased
the performance efficiency index, as well as the actual runtime reduction.  

\section*{Performance Metrics}
We will primarily be gauging our performance by using the Performance Efficiency Index
(PEI) in order to measure query execution times against differing data partitioning
strategies, memory management strategies, as well as varying degrees of parallel execution.
The PEI is simply database time, the time a database process spends executing code, divided
by wall clock time which is time elapsed in the real world. In addition to PEI, we can
also measure CPU usage, memory usage, disk input and output, and the paths taken within
the database to access data. From a user’s perspective, wall clock time is going to be
one of the most important metrics since it will determine the perceived responsiveness
of queries made against the database

\end{document}