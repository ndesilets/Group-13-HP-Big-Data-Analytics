\documentclass[draftclsnofoot, onecolumn, compsoc, 10pt]{IEEEtran}

\title{\Huge Preliminary Design Document\\\large HP: Big Data Analytics\\Group 13}

\author{Oregon State University\\CS 461\\2016-2017\\\\Prepared By:\\Nic Desilets, James Stallkamp,\\and Nathaniel Whitlock}

\usepackage{titling}

%\usepackage{hyperref}

\usepackage[margin=0.75in]{geometry}

\usepackage{graphicx}

\linespread{1.0} % 1.0 = single, 1.3 = 1.5, 1.6 = double

\parindent=0.0in

\parskip=0.2in

\begin{document}
\begin{titlingpage}
    \maketitle 
    
    \vspace{1in}
    \begin{abstract}
		\noindent This document is focuses on a detailed brake down of "how" the individual components of our project will be implemented.
        Each of the three essential pieces: SQL development environment; parallelization and partitioning; and reporting tools are covered with fine granularity.\\\\
        \textbf{Elaborate more here in the future...}
        
    \end{abstract}
\end{titlingpage}

%\tableofcontents
{\small\tableofcontents} % Reduce the size of TOC to make it 1 page
\addtocontents{toc}{\protect\thispagestyle{empty}} % Removes page number from TOC

% Add content here
\section{Introduction}
This should include the full *HOW* of your system.
This includes your API, your timeline, necessary testing information, etc.
This is important for you as a guide. Remember, a week of debugging can save you 20 minutes of planning! Plan the work, work the plan.
For research projects, this will look a little different. This will be more on the order of "experiment design", where you detail the junction points/milestones. You will also detail what will dictate your path from then on. You should provide an expected path through these points, any algorithms you will be using or designing, etc.

% James' Section
\section{SQL Development Environment}
\textbf{Miles Stones for getting IDE set up.}
\begin{itemize} 
	\item Install local Oracle Instance
    \begin{itemize}
    	\item Download and install Virtual Box.
        \item Download and mount Oracle 12c x64 iso.
        \item Run Oracle 12c preinstaller.
        \item Set up Oracle 12 server on Virtual Box.
    \end{itemize}
    \item Install SQL developer
    \begin{itemize}
    	\item Download and install SQL developer.
        \item Connect to local server.
        \item See next step for connecting to HP test server. 
    \end{itemize}
    \item Connect to HP blade server(on campus)
    \begin{itemize}
    	\item Must be on HP campus to connect to test server.
        \item Once connected to HP server ready to run experiments. 
    \end{itemize}
\end{itemize}
% Nate's Section
\section{Parallelization and Partitioning}

\textbf{List of Potential Ideas}
\begin{itemize} 
	\item  Method of modifying setting for both techniques
    \begin{itemize} 
		\item Degree of Parallelism (DOP) 
        \item Number of independent processes available to PX coordinator
        \item Cardinality and optimizer statistics
    \end{itemize}
    \item Method of visually evaluating the performance difference
    \begin{itemize} 
		\item Important to poster/expo
        \item Discuss how the raw data will be made flexible for multiple visualization techniques
        \item Discuss what visualizations will be used (or most reasonable for our purpose)
        \item Possibly talk about how we will know when we have met the ideal treshold for system performance or data return
    \end{itemize}
    \item Partition Designs
    \begin{itemize} 
		\item Evaluate the ideal case for the use of each partition design in order to develop an intuition for selecting the ideal choice
        \item Break down some of the reporting analytic queries in order to see how the data is being stored (ie. do they need the entire row, or just some values?)
        \item Develop use case for each partition design in terms of the reporting queries that HP runs
    \end{itemize}
    \item Using toolkit to adjust settings, enable parallel queries with a "hint"
    \begin{itemize}
    	\item Talk about what a hint is
        \item Figure out how it is possible to configure px settings through SQL statements
    \end{itemize}
\end{itemize}

% Nic's Section
\section{Reporting Tools}
\textbf{List of Potential Ideas}
\begin{itemize} 
	\item Collection of Raw Data
    \begin{itemize}
    	\item List the main v\$ views that we will query for session statistics (ie. v\$active\_session\_history)
        \item Identify how we can correctly calculate statistics of interest (DB time, I/O, etc...)
        \item How we would focus development of homebrew toolkit to expand on what is available within the other options
    \end{itemize}
    \item Milestones
\end{itemize}

\section{Conclusion}
\end{document}